\section{Numerical simulations}
\label{sec:numerical-simulations}

\subsection{Flops count}
\label{subsec:flops-count}

Figure \ref{fig:flops} shows the total number of flops for solving the overdetermined problem
(equation \ref{eq:delta-q-tilde-overdetermined}) with different equivalent-layer methods
(equations \ref{flops:cholesky}, \ref{flops:cgls}, \ref{flops:LS89}, \ref{flops:SU21}, 
\ref{flops:C92}, \ref{flops:reparameterization-cgls}, \ref{flops:SOB17}, \ref{flops:TOB20},
and \ref{flops:direct-deconv}), by considering 
the particular case in which $\mathbf{H} = \mathbf{I}_{P}$ (equation \ref{eq:reparameterization} and 
subsection \ref{subsec:formulation-without-reparameterization}),
$\mu = 0$ (equation \ref{eq:function-Gamma}), 
$\mathbf{W}_{d} = \mathbf{I}_{D}$ (equation \ref{eq:function-Phi}) and
$\bar{\mathbf{p}} = \mathbf{0}$ (equation \ref{eq:reparameterization-reference}), 
where $\mathbf{I}_{P}$ and $\mathbf{I}_{D}$ are the identities of order $P$ and $D$, respectively.
The flops are computed for different number of potential-field data ranging from $10,000$ to $1,000,000$.
Figure \ref{fig:flops} shows that the moving data-window strategy by using 
\citeauthor{leao-silva1989}'s \citeyear{leao-silva1989} method and direct deconvolution are the fastest methods.

The control parameters to run the equivalent-layer methods shown in Figure 
 \ref{fig:flops}  are the following: 
i) in CGLS, reparameterization approaches \cite[e.g.,][]{oliveirajr-etal2013, mendonca2020}, \cite{siqueira-etal2017}, and 
\cite{takahashi-etal2020} (equations \ref{flops:cgls} \ref{flops:C92}), 
\ref{flops:reparameterization-cgls},  \ref{flops:SOB17} and \ref{flops:TOB20}) we set $\mathtt{ITMAX} = 50$; 
ii) \cite{cordell1992} we set ITMAX = $6 \, D$; 
iii) in \cite{leao-silva1989} (equation \ref{flops:LS89}) we set 
$D'= 49 \: (7 \times 7)$ and $P' = 225 \: (15 \times 15) $; and
iv) in \cite{soler-uieda2021} (equation  \ref{flops:SU21}) we set 
$D'= P' = 900 \: (30 \times 30)$.

\subsection{Synthetic potential-field data}
\label{subsec:synthetic-data}

We create a model composed of several rectangular prisms that can be split into three groups.
The first is composed of $300$ small cubes (not shown) with top at $0 \: \mathrm{m}$ and side lengths
defined according to a pseudo-random variable having uniform distribution from 
$100$ to $200 \: \mathrm{m}$.
Their density contrasts are defined by a pseudo-random variable uniformly distributed
from $1000$ to $2000 \: \mathrm{kg / m^{3}}$.
These prisms produce the short-wavelength component of the simulated gravity data.
The $4$ prisms forming the second group of our model (indicated by A-D in Figure \ref{fig:synthetic-bodies}) 
have tops varying from $10$ to $100 \: \mathrm{m}$ and bottom from $1010$ to $1500 \: \mathrm{m}$.
They have density contrasts of $1500$, $-1800$, $-3000$ and $1200  \: \mathrm{kg / m^{3}}$
and side lengths varying from $1000$ to $4000 \: \mathrm{m}$.
These prisms produce the mid-wavelength component of the simulated gravity data.
There is also a single prism (indicated by E in Figure \ref{fig:synthetic-bodies}) 
with top at $1000 \: \mathrm{m}$, bottom at 
$1500 \: \mathrm{m}$ and side lengths of $4000$ and $6000 \: \mathrm{m}$.
This prism has density contrast is $-900 \: \mathrm{kg / m^{3}}$ and produces the
long-wavelength of our synthetic gravity data.

We have computed noise-free gravity disturbance and gravity-gradient tensor components 
produced by our model on a regularly spaced grid of $50 \times 50$ points at 
$z = -100 \: \mathrm{m}$ (Figure \ref{fig:noise-free-data}).
We have also simulated additional $L = 20$ gravity disturbance data sets $\mathbf{d}^{\ell}$, $\ell \in \{1:L\}$, 
by adding pseudo-random Gaussian noise with zero mean and crescent standard deviations to the noise-free data (not shown).
The standard deviations vary from $0.5\%$ to $10\%$ of the maximum absolute value in the noise-free data,
which corresponds to $0.21$ and $4.16 \: \mathrm{mGal}$, respectively.

\subsection{Stability analysis and gravity-gradient components}

We set a planar equivalent layer of point masses having one source below each datum at a constant vertical coordinate $z \approx 512.24 \: \mathrm{m}$.
This depth was set by following the \citeauthor{dampney1969}'s (\citeyear{dampney1969}) criterion (see Subsection \ref{subsec:spatial-distribution-sources}),
so that the vertical distance $\Delta z$ between equivalent sources and the simulated data is equal to $3 \times$ the grid spacing 
($\Delta x = \Delta y \approx 204.08 \: \mathrm{m} $).
Note that, in this case, the layer has a number of sources $P$ equal to the number of data $D$.

We have applied the Cholesky factorization (equations \ref{eq:Cholesky-factorization} and \ref{eq:Cholesky-solver}), 
CGLS (Algorithm \ref{alg:CGLS}), the iterative method of \citet{siqueira-etal2017} (Algorithm \ref{alg:SOB17}), 
the iterative deconvolution (Algorithms \ref{alg:TOB20-22} and \ref{alg:fast-2D-convolution}) proposed by 
\citet{takahashi-etal2020} (Algorithm \ref{alg:TOB20-22}) and the direct deconvolution 
(equations \ref{eq:direct-deconvolution} and \ref{eq:matrix-L-Wiener-deconvolution})
with four different values for the parameter $\zeta$ to the $21$ gravity data sets.

For each method, we have obtained one estimate $\tilde{\mathbf{p}}$ from the noise-free gravity data $\mathbf{d}$
and $L=20$ estimates $\tilde{\mathbf{p}}^{\ell}$ from the noise-corrupted gravity data $\mathbf{d}^{\ell}$, $\ell \in \{1:L\}$,
for the planar equivalent layer of point masses, totaling $21$ estimated parameter vectors and 
$20$ pairs $\left( \Delta p^{\ell} \: , \: \Delta d^{\ell} \right)$ of model and data perturbations
(equations \ref{eq:model-perturbation} and \ref{eq:data-perturbation}).
Figure \ref{fig:stability-comparison} shows the numerical stability curves computed with each method for 
the synthetic gravity data.

All these $21$ estimated parameters vectors were obtained by solving 
the overdetermined problem (equation \ref{eq:delta-q-tilde-overdetermined}) with the same method for the particular case in which
$\mathbf{H} = \mathbf{I}$ (equation \ref{eq:reparameterization} and 
subsection \ref{subsec:formulation-without-reparameterization}),
$\mathbf{W}_{d} = \mathbf{W}_{q} = \mathbf{I}$ (equations \ref{eq:function-Phi} and \ref{eq:function-Theta}) and
$\bar{\mathbf{p}} = \mathbf{0}$ (equation \ref{eq:reparameterization-reference}), where $\mathbf{I}$ is the identity of order $D$.

Figure \ref{fig:stability-comparison} shows how the numerical stability curves vary as the level of the noise is increased. 
We can see that for all methods, a linear tendency is observed as it is expected. The inclination of the straight line indicates the stability of each method. 
As shown in Figure \ref{fig:stability-comparison}, the direct deconvolution with $\zeta = 0$ exhibits a high slope, which indicates high instability and 
emphasizes the necessity of using the Wiener filter ($\zeta > 0$ in equation \ref{eq:matrix-L-Wiener-deconvolution}). 

The estimated stability parameters $\kappa$ (equation \ref{eq:condition-number}) obtained for the Cholesky factorization, CGLS and
iterative deconvolution are close to each other (Figures \ref{fig:stability-comparison}).
They are slightly smaller than that obtained for the iterative method of \citet{siqueira-etal2017} ($\mathtt{SOB17}$).
Note that by varying the parameter $\zeta$ (equation \ref{eq:matrix-L-Wiener-deconvolution}) it is possible to obtain different 
stability parameters $\kappa$ for the direct deconvolution. There is no apparent rule to set $\zeta$.
A practical criterion can be the maximum $\zeta$ producing a satisfactory data fit. Overshoot values tend to exaggeratedly smooth the 
predicted data. 

We inverted the noise-corrupted gravity disturbance  with the highest noise level (not shown) 
to estimate an equivalent layer (not shown) via iterative deconvolution (Algorithm \ref{alg:TOB20-22}).
Figure \ref{fig:residuals-TOB20}\textbf{(G)} shows the residuals (in mGal) between the predicted and noise-corrupted 
gravity disturbances.
As we can see, the residuals are uniformly distributed on simulated area and suggest that the equivalent layer
produces a good data fit.
This can be verified by inspecting the histogram of the residuals between the predicted and noise-corrupted 
gravity disturbances shown in panel \textbf{(G)} of Figure \ref{fig:histograms-TOB20}.

Using the estimated layer, we have computed the gravity-gradient data (not shown) at the observations points.
Figures \ref{fig:residuals-TOB20} \textbf{(A)}--\textbf{(F)} show the residuals (in Eötvös)
between the predicted (not shown) and noise-free gravity-gradient data 
(Figure \ref{fig:noise-free-data}).
These figures show that the iterative deconvolution (Algorithm \ref{alg:TOB20-22}) could predict the six components of the
gravity-gradient tensor with a good precision, which can also be verified in the corresponding histograms shown in 
Figure \ref{fig:histograms-TOB20}.

In the supplementary material, we show the residuals between the gravity data predicted by the equivalent layer estimated by using the following methods:
i)  the CGLS method (Algorithm \ref{alg:CGLS});
ii) the  Cholesky factorization (equations \ref{eq:Cholesky-factorization} and \ref{eq:Cholesky-solver});
iii) the iterative method proposed by \citet{siqueira-etal2017} (Algorithm \ref{alg:SOB17});
iv) the direct deconvolution with  optimal value of $\zeta = 10^{-22}$ 
(equation \ref{eq:matrix-L-Wiener-deconvolution}); and
v) the iterative method  proposed by \cite{cordell1992} (Algorithm \ref{alg:C92}).



%Gravity synthetic statistics to be included:
%Means
%0.24791339230971493 (Classical method)
%0.25522040542133817 (CG BTTB method)
%0.86010282709889 (Deconvolutional method)
%1.53835137193657 (Deconvolutional w\ Wiener overshoot $\mu$ method)
%0.3134732823974472 (Deconvolutional w\ Wiener optimal $\mu$ method)
%0.5553048046997608 (Deconvolutional w\ Wiener suboptimal $\mu$ method)
%
%Standard deviations
%0.18274083156485463 (Classical method)
%0.18986126212291252 (CG BTTB method)
%1.439293452270024 (Deconvolutional method)
%1.1183051446613188 (Deconvolutional w\ Wiener overshoot $\mu$ method)
%0.2367045031838225 (Deconvolutional w\ Wiener optimal $\mu$ method)
%0.7047326489645682 (Deconvolutional w\ Wiener suboptimal $\mu$ method)
%\\\\
%Magnetic synthetic statistics to be included:
%
%Means
%
%6.572366904728528 (Classical method)
%6.689214592343857 (CG BTTB method)
%971.9310697001104 (Deconvolutional method)
%23.672356138290855 (Deconvolutional w\ Wiener overshoot μ method)
%8.09848247511561 (Deconvolutional w\ Wiener optimal μ method)
%29.264824940161848 (Deconvolutional w\ Wiener suboptimal μ method)
%
%standard deviations
%4.9276131862044315 (Classical method)
%4.9909561696899205 (CG BTTB method)
%1742.0801705255908 (Deconvolutional method)
%17.671099090589646 (Deconvolutional w\ Wiener overshoot μ method)
%6.1358737706945075 (Deconvolutional w\ Wiener optimal μ method)
%29.374157656258866 (Deconvolutional w\ Wiener suboptimal μ method)