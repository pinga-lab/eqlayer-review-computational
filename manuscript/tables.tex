\section{Tables}

\begin{table}[h!]
	\centering
	\begin{tabular}{c c c}
		Reference & Term & flops \\
		\hline
		eq. \ref{eq:predicted-data-vetor-reparameterized} & $\mathbf{G \, H}$ & $2DQP$ \\
		eq. \ref{eq:vector-p-tilde} & $\mathbf{H} \, \tilde{\mathbf{q}}$ & $2PQ$ \\
		eq. \ref{eq:delta-q-tilde-overdetermined} & $\left(\mathbf{G \, H}\right)^{\top}\left(\mathbf{G \, H}\right)$ & $2Q^{2}D$ \\
		eq. \ref{eq:delta-q-tilde-overdetermined} & $\left(\mathbf{G \, H}\right)^{\top}\boldsymbol{\delta}_{d} $ & $2QD$ \\
		eq. \ref{eq:delta-q-tilde-underdetermined} & $\left(\mathbf{G \, H}\right) \, \left(\mathbf{G \, H}\right)^{\top}$ & $2D^{2}Q$ \\
		eq. \ref{eq:delta-q-tilde-underdetermined} & $\left(\mathbf{G \, H}\right)^{\top}\mathbf{u}$ & $2QD$ \\
		eq. \ref{eq:Cholesky-factorization} & lower triangle of $\boldsymbol{\mathcal{G}}$ & $\nicefrac{D^{3}}{3}$ or $\nicefrac{Q^{3}}{3}$ \\
		eq. \ref{eq:Cholesky-solver} & solve triangular systems & $2D^{2}$ or $2Q^{2}$ \\
		Alg. \ref{alg:CGLS} & $\boldsymbol{\eta} \gets \boldsymbol{\vartheta} + \tau \, \boldsymbol{\eta}$ & 2Q \\
		Alg. \ref{alg:CGLS} & $\boldsymbol{\vartheta}^{\top} \boldsymbol{\vartheta}$ & 2Q \\
		Alg. \ref{alg:SOB17} & $\boldsymbol{\sigma} \circ \mathbf{d}$ & $D$ \\
	\end{tabular}
	\caption{
		Total number of flops associated with some useful terms according to \citet[][p. 12]{golub-vanloan2013}.
		The number of flops associated with equations \ref{eq:Cholesky-factorization} and \ref{eq:Cholesky-solver} 
		depends if the problem is over or underdetermined.
		Note that $P = Q$ for the case in which $\mathbf{H} = \mathbf{I}_{P}$ (subsection \ref{subsec:formulation-without-reparameterization}).
		The term associated with Algorithm \ref{alg:CGLS} is a vector update called \textit{saxpy} \citep[][p. 4]{golub-vanloan2013}.
		The terms defined here are references to compute the total number of flops throughout the manuscript.
	}
	\label{tab:standard-flops}
\end{table}

\newpage


%\begin{landscape}
\begin{table}[h!]
%\begin{sidewaystable}[h!] 
	\centering
	\begin{tabular}{c c c c c}
	
		\thead{Computational \\ strategies} & Characteristics & Advantages & Disadvantages  & articles \\
		\hline
		\\
		\thead{Moving data-window \\ scheme} & 
		\thead{A single  and small \\ 
		sensitivity submatrix \\ for all moving windows} &
		\thead{One of the fastest \\ strategies}  &
		\thead{Regularly spaced grids \\ of sources and data} & 
		\thead{\cite{leao-silva1989}} \\ 
		
		\thead{Moving data-window \\ scheme} & 
		\thead{Multiple and small \\ 
		sensitivity submatrices, \\ one for each moving}  & 
		\thead{Irregularly spaced grids \\ of sources and data} & 
		\thead{Computational speed \\ is reduced} & 
		\thead{\cite{soler-uieda2021}} \\ \\
		
		\thead{Column-action \\ updates} & 
		\thead{A single equivalent\\ source is used, \\ iteratively} &
		\thead{A single column \\ of the sensitivity matrix \\ 
		is 	calculated}  & 
		\thead{Issues related to \\ convergence} & 
		\thead{\cite{cordell1992} \\ \cite{guspi-novara2009}} \\ 
		
				
		\thead{Row-action \\ updates} & 
		\thead{Equivalent data concept} &
		\thead{A subset of rows \\ of the sensitivity matrix \\ 
		is 	calculated}  & 
		\thead{Increasing the order \\ of the linear system  \\
		of 	equations, iteratively} & 
		\thead{\cite{mendonca-silva1994}} \\ \\ 
			
		\thead{Reparametrization of \\ the  original parameters} & 
		\thead{Reduction the dimension \\ of the linear system \\ 
		of 	equations} &
		\thead{Lower-dimensional \\ linear system \\  of equations}  & 
		\thead{Undesirable \\ smoothing effect} & 
		\thead{\cite{oliveirajr-etal2013} \\ \cite{mendonca2020}} \\ \\ 

		\thead{Sparsity induction of \\ the sensitivity matrix} & 
		\thead{Sparse representation \\  of the original dense \\ 
		sensitivity matrix} &
		\thead{Fast iteration \\ of the CG algorithm}  & 
		\thead{Requires computing \\  the full and dense \\ 
		sensitivity matrix} & 
		\thead{\cite{li-oldenburg2010} \\ \cite{barnes-lumley2011}} \\ \\ 

		\thead{Iterative methods using the \\ full sensitivity matrix} & 
		\thead{The equivalent layer is \\ updated, iteratively} &
		\thead{Fast iterations}  & 
		\thead{Requires computing \\  the full and dense \\ 
		sensitivity matrix} & 
		\thead{\cite{xia-sprowl1991} \\ \cite{xia-etal1993} \\
		\cite{siqueira-etal2017} \\ 	\cite{jirigalatu-ebbing2019}	} \\ \\
		
		\thead{Iterative deconvolution } & 
		\thead{Block-Toeplitz \\ Toeplitz-block (BTTB) \\ 
		matrices concept} &
		\thead{One of the fastest \\ strategies}  & 
		\thead{Regularly spaced grids \\ of sources and data} & 
		\thead{\cite{takahashi-etal2020} 
		\\ \cite{takahashi-etal2022}} \\ \\

		\thead{Direct deconvolution } & 
		\thead{BTTB	matrices concept} &
		\thead{One of the fastest \\ strategies}  & 
		\thead{Solution instability} & 
		\thead{ } \\ \\



 		\end{tabular}
	
	\caption{Computational strategies to overcome the intensive computational cost of the
       equivalent-layer technique for processing potential-field data and the corresponding 	
       articles.
	}
	\label{tab:discussion}
\end{table}
%\end{sidewaystable}