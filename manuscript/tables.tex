\section{Tables}

\begin{table}[h!]
	\centering
	\begin{tabular}{c c c}
		Reference & Term & flops \\
		\hline
		eq. \ref{eq:predicted-data-vetor-reparameterized} & $\mathbf{G \, H}$ & $2DQP$ \\
		eq. \ref{eq:vector-p-tilde} & $\mathbf{H} \, \tilde{\mathbf{q}}$ & $2PQ$ \\
		eq. \ref{eq:delta-q-tilde-overdetermined} & $\left(\mathbf{G \, H}\right)^{\top}\left(\mathbf{G \, H}\right)$ & $2Q^{2}D$ \\
		eq. \ref{eq:delta-q-tilde-overdetermined} & $\left(\mathbf{G \, H}\right)^{\top}\boldsymbol{\delta}_{d} $ & $2QD$ \\
		eq. \ref{eq:delta-q-tilde-underdetermined} & $\left(\mathbf{G \, H}\right) \, \left(\mathbf{G \, H}\right)^{\top}$ & $2D^{2}Q$ \\
		eq. \ref{eq:delta-q-tilde-underdetermined} & $\left(\mathbf{G \, H}\right)^{\top}\mathbf{u}$ & $2QD$ \\
		eq. \ref{eq:Cholesky-factorization} & lower triangle of $\boldsymbol{\mathcal{G}}$ & $\nicefrac{D^{3}}{3}$ or $\nicefrac{Q^{3}}{3}$ \\
		eq. \ref{eq:Cholesky-solver} & solve triangular systems & $2D^{2}$ or $2Q^{2}$ \\
		Alg. \ref{alg:CGLS} & $\boldsymbol{\eta} \gets \boldsymbol{\vartheta} + \tau \, \boldsymbol{\eta}$ & 2Q \\
		Alg. \ref{alg:CGLS} & $\boldsymbol{\vartheta}^{\top} \boldsymbol{\vartheta}$ & 2Q \\
		Alg. \ref{alg:SOB17} & $\boldsymbol{\sigma} \circ \mathbf{d}$ & $D$ \\
	\end{tabular}
	\caption{
		Total number of flops associated with some useful terms according to \citet[][p. 12]{golub-vanloan2013}.
		The number of flops associated with equations \ref{eq:Cholesky-factorization} and \ref{eq:Cholesky-solver} 
		depends if the problem is over or underdetermined.
		Note that $P = Q$ for the case in which $\mathbf{H} = \mathbf{I}_{P}$ (subsection \ref{subsec:formulation-without-reparameterization}).
		The term associated with Algorithm \ref{alg:CGLS} is a vector update called \textit{saxpy} \citep[][p. 4]{golub-vanloan2013}.
		The terms defined here are references to compute the total number of flops throughout the manuscript.
	}
	\label{tab:standard-flops}
\end{table}

\newpage

%\begin{landscape}
\begin{table}[h!]
	\centering
	\begin{tabular}{c c c c c}
	
		Computational strategies & Characteristics & Advantages & Disadvantages  & articles \\
		\hline
		\\
		Moving data-window scheme & 
		A single sensitivity submatrix for all moving windows & 
		One of the fastest strategies  &
		Regularly spaced grids of sources and data & 
		\cite{leao-silva1989} \\
		
		Moving data-window scheme & 
		Multiple sensitivity submatrices, one for each moving  & 
		Irregularly spaced grids of sources and data & 
		Computational speed is reduced & 
		\cite{soler-uieda2021} \\
			\end{tabular}
	
	\caption{Computational strategies to overcome the intensive computational cost of the
       equivalent-layer technique for processing potential-field data and the corresponding 	
       articles.
	}
	\label{tab:discussion}
\end{table}
%\end{landscape}