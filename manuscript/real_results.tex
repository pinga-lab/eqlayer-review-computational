\section{Applications to field data}
\label{sec:real_data}

In this section, we show the applications of the iterative (Algorithm \ref{alg:TOB20-22}) and direct 
(equations \ref{eq:direct-deconvolution} and \ref{eq:matrix-L-Wiener-deconvolution}) deconvolutions to field data sets 
over the Caraj{\'a}s Mineral Province (CMP) in the Amazon craton \citep{moroni-etal2001,villas-santos2001}. 
This area is known for its intensive mineral exploration such as iron, copper, gold, manganese, and, recently, bauxite.

\subsection{Geological setting}

The Amazon Craton is one of the largest and least-known Archean-Proterozoic areas in the world, comprehending a region with a thousand square kilometers. 
It is one of the main tectonic units in South America, which is covered by five Phanerozoic basins: 
Maranh{\~a}o (Northeast), Amazon (Central), Xingu-Alto Tapaj{\'o}s (South), Parecis (Southwest), and Solim{\~o}es (West). 
The Craton is limited by the Andean Orogenic Belt to the west and the by Araguaia Fold Belt to the east and southeast. 
The Amazon craton has been subdivided into provinces according to two models, one geochronological and the other geophysical-structural 
\citep{amaral1974, teixeira-etal1989, tassinari-macambira1999}. Thus, seven geological provinces with distinctive ages, evolution, and structural 
patterns can be observed, namely: (i) Caraj{\'a}s with two domains - the Mesoarchean Rio Maria and Neoarchean Caraj{\'a}s; 
(ii) Archean-Paleoproterozoic Central Amazon, with Iriri-Xingu and Curuá-Mapuera domains; (ii) Trans-Amazonian (Ryacian), with the Amap{\'a} 
and Bacaj{\'a} domains; (iv) the Orosinian Tapaj{\'o}s-Parima, with Peixoto de Azevedo, Tapaj{\'o}s, Uaimiri, and Parima domains; 
(v) Rond{\^o}nia-Juruena (Statherian), with Jamari, Juruena, and Jauru domains; (vi) The Statherian Rio Negro, with Rio Negro and 
Imeri domains; and (vii) Suns{\'a}s (Meso-Neoproterozoic), with Santa Helena and Nova Brasil{\^a}ndia domains \citep{santos-etal2000}. 
Nevertheless, we focus this work only on the Caraj{\'a}s Province. 

The Caraj{\'a}s Mineral Province (CMP) is located in the east-southeast region of the craton, within an old tectonically stable nucleus in the 
South American Plate that became tectonically stable at the beginning of Neoproterozoic \citep{salomao-etal2019}. 
This area has been the target of intensive exploration at least since the final of the '60s, after the discovery of large iron ore deposits. 
There are several greenstone belts in the region, among them are the Andorinhas, Inajá, Cumaru, Caraj{\'a}s, Serra Leste, Serra Pelada, and Sapucaia 
\citep{santos-etal2000}. The mineralogic and petrologic studies in granite stocks show a variety of minerals found in the province, such as 
amphibole, plagioclase, biotite, ilmenite, and magnetite \citep{cunha-etal2016}. 

\subsection{Potential-field data}

The field data used here were obtained from an airborne survey conducted by Lasa Prospec{\c c}{\~o}es S/A. 
and Microsurvey Aerogeof{\'i}sica Consultoria Cient{\'i}fica Ltda. between April/2013 and October/2014.
The survey area covers $\approx 58000 \: \mathrm{km}^2$ between latitudes $-8^{\circ} -- -5^{\circ}$
and longitudes $-53^{\circ} -- -49.5^{\circ}$ referred to the WGS-84 datum.
We obtained the horizontal coordinates $x$ and $y$ already in the UTM zone 22S. 
The flight and tie lines are spaced at $3 \: \mathrm{km}$  and $12 \: \mathrm{km}$, with orientation along 
directions $N-S$ and $E-W$, respectively.
The data are placed at an approximately constant distance of $900 \: \mathrm{m}$ above the ground.
Figure \ref{fig:carajas-data} shows the $D = 500,000$ aerogravimetric data on a grid of 
$1000 \times 500$ observation points with $\Delta x = 358.12 \: \mathrm{m}$ and $\Delta y = 787.62 \: \mathrm{m}$. 

\subsection{Potential-field transformation}

We applied the equivalent-layer technique to the observed data (Figure \ref{fig:carajas-data}) with the
purpose of illustrating how to estimate the gravity-gradient tensor over the study area.
We used an equivalent layer layout with one source located below each datum
(so that $P=D$) on a horizontal plane having a vertical distance $\Delta z \approx 2362.86 \: \mathrm{m}$ from the observation plane.
This setup is defined by setting $\Delta z \approx 3 \, dy$, which follows the same strategy of \citet{reis-etal2020}.
We solve the linear inverse problem for estimating the physical-property distribution on the layer by
using the iterative deconvolution (Algorithm \ref{alg:TOB20-22}) with a maximum number of $50$ iterations. 
Actually, the algorithm have converged with only $18$ iterations.

Figure \ref{fig:carajas-grav-gradient}\textbf{(G)} shows the histogram of the residuals between the predicted 
(not shown) and observed data (Figure \ref{fig:carajas-data}).
As we can see, the iterative deconvolution produced an excellent data fit.
By using the estimated layer, we have computed the gravity-gradient tensor components at the observation points.
The results are shown in Figures \ref{fig:carajas-grav-gradient}\textbf{(A)}--\textbf{(F)}.

Considering the processing time, the iterative deconvolution took  $\approx 1.98 \: \mathrm{s}$ to execute the $18$
iterations for estimating the physical-property distribution on the layer by inverting the $D = 500,000$ observed data.
The code was run in a modest computer with $16,0 \: \mathrm{GiB}$ of memory and processor 
$\mathrm{12th \, Gen \, Intel\textregistered \, Core\texttrademark \, i9-12900H \times 20}$.
Given the estimated equivalent layer, the gravity-gradient components shown in Figure \ref{fig:carajas-grav-gradient} were computed in
$\approx 0.52 \: \mathrm{s}$.
These results demonstrate the efficiency of the iterative deconvolution method in processing large datasets.


%\section{Real data application}
%\label{sec:real_data}

%Gridded data of 1000x500 (500000 observed points) for both grav and mag. Data is at -900m.

%Grav equivalent layer depth is 300 m and 50 iterations of the cgls method was used.
%Mag equivalent layer depth is 0 m and 200 iterations of the cgls method was used.

%On an Intel Core i7 7700HQ@2.8 GHz processor in single processing and single-threading modes
%the gravimetric equivalent layer took $9.19$ seconds to estimate the equivalent sources with the convolutional method and $0.51$ seconds with the deconvolutional method. 
%
%The magnetic equivalent layer took $82$ seconds to estimate the equivalent sources with the convolutional method and $0.84$ seconds with the deconvolutional method.
%
%As Carajás area is very large different values of the magnetic main field can be considered. 
%The main field declination was calculated using the tool in the website (for the date 01/01/2014): https://www.ngdc.noaa.gov/geomag/calculators/magcalc.shtml
%For this application I considered an approximated mid location of the area (latitude $-6.55^{\circ}$ and longitude $-50.75^{\circ})$. The declination is $-19.865^{\circ}$ and the inclination $-7.43915^{\circ}$.
%As the source magnetization is unknown inclination and declination equal to the main field is being used for all the equivalent sources.
%
%Gravimetric case:
%
%Means
%
%0.0005096975472675431 (convolutional method)
%
%0.4582999511463665 (deconvolutional method with wiener $\mu = 10^{-22}$)
%
%Standart deviations
%
%0.15492798729938298 (convolutional method)
%
%1.229507199000529 (deconvolutional method with wiener $\mu = 10^{-22}$)
%\\\\
%Magnetic case:
%
%Means
%
%-0.06404347121632468 (convolutional method)
%
%18.992921718679344 (deconvolutional method with wiener $\mu = 10^{-16}$)
%
%Standart deviations
%
%1.9687559764381535 (convolutional method)
%
%33.641199020925924 (deconvolutional method with wiener $\mu = 10^{-16}$)